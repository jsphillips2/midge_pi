\section*{Introduction}

While there are many parameterizations of PI, 
one that has been particularly popular in recent work is of the following form:
%
\begin{equation} \label{eq:pi}
  \text{GPP} = 
    \beta \times \text{tanh}\left(\frac{\alpha} {\beta} \times \text{PAR}\right)
\end{equation}
%
where $\text{PAR}$ is the photosynthetically active irradiance,
$\beta$ is the maximum GPP attained under conditions where light is not limiting,
and $\alpha$ is the sensitivity of GPP to increases in light when light is limiting.
One advantage of this parameterization is that the parameters have a direct 
biological interpretation. 
The maximum of the PI curve represents the peak photosynthetic potential,
which increases with factors such as standing biomass and
availability of resources that become limiting when light is in excess (e.g., nutrients).
In contrast, the sensitivity of GPP to increases in light represents the ability 
of primary producers to exploit the available light when it is the primary limiting factor,
which increases with factors such as the concentration of photosynthetic pigments or the 
geometry of the substrate.
Despite their different interpretations, increases in $\beta$ and $\alpha$ both increase
GPP at a given light level.
Furthermore, they both have the potential to increase in response to 
ecosystem engineering.
For example, substrate structurring provided by benthic invertebrates can increase
the surface area for biofilm growth, 
thereby ameliorating competition for light and increasing the photosynthetic 
biomass under high light conditions.

While equation (\ref{eq:pi}) makes clear the influence of different aspects of photosynthesis
on the shape of the curve, 
another important consideration is how the shape of this curve relates to ambient light
conditions, which is captured in the concept of "saturation". 
Often this is expressed as the half-saturation level 
(i.e., the denominator constant in the classical Michaelis-Menten curve), 
which can be inferred from (\ref{eq:pi}) as
%
\begin{equation} \label{eq:sat}
  \text{PAR}_{\text{half~sat}} = 
    \frac{\beta} {\alpha} \times \text{tanh}^{-1} \left(\frac{1}{2}\right)
\end{equation}
%
Equation \ref{eq:sat} shows that $\beta$ and $\alpha$ have opposite effects on the 
half-saturation level, 
despite having coindiciding effects on overall GPP.
This in turn means that variations in $\beta$ and $\alpha$ have 
different implications for how primary production responds to spatiotemporal variation 
in light conditions.
Therefore,
understanding the effects of aquatic ecosystem engineers on benthic primary production
requires the separation of their potential effects on different aspects of the PI curve.

