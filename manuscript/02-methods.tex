

\section*{Methods}

I fit a photosynthesis-irradiance curve to of the following form:
%
\begin{equation} \label{eq:fit}
  \Delta \text{DO}_i = 
    \beta_i \times \text{tanh}\left(\frac{\alpha_i} {\beta_i} \times \text{PAR}\right)
      - \rho_i + \epsilon_i
\end{equation}
%
where $\Delta \text{DO}_i$ is the hourly change in dissovled oxygen 
in experimental unit $i$, 
$\beta_i$ and $\alpha_i$ are parameters of the PI curve as defined in equation \ref{eq:pi},
$\rho_i$ is total respiration of organisms in the sediment,
and $\epsilon_i$ is the residual error.
I modeled the effects of sediment treatment and experimental day as
%
\begin{equation} \label{eq:lm}
\begin{split}
\beta_i &\sim 1 + \text{day} \times \text{treatment} \\
\alpha_i &\sim 1 + \text{day} \times \text{treatment} \\
\rho_i &\sim 1 + \text{day} \times \text{treatment}
\end{split}
\end{equation}
%
using the "formula" notation for linear models, 
where `1' denotes the intercept,
and `$\times$' denotes both main effects and interactions.

To assess ecosystem engineering effects on different aspects photosynthesis, 
I compared the full model reduced versions that omitted main effects and interactions from
the formulas of either $\beta_i$ and $\alpha_i$.
I removed main effects and interactions involving the sediment treatments together
both to reduce the number of models and because of the a prior expection that 
the treamtent effects is unlikely to remain constant through time.
Simiarly, the main effects of interactions of day and sediment treatment on $\rho_i$
were included in all models to reduce the number of alternatives and because 
Phillips et al. (2019) found strong evidence for both the main effects and interactions on
$\rho_i$.
I compared the alternative models using the information criterion associated 
with Leave-one-out cross validation (``LOOIC''), 
which is conceptually similar to the  Akaike Information Criterion (AIC).
The models were fit using Stan run from R using the ``rstan'' package.

