

\section*{Methods}

I fit a PI curve of the following form:
%
\begin{equation} \label{eq:fit}
  \Delta \text{DO}_i = 
    \beta_i \times \text{tanh}\left(\frac{\alpha_i} {\beta_i} \times \text{PAR}_i\right)
      - \rho_i + \epsilon_i
\end{equation}
%
where $\Delta \text{DO}_i$ is the hourly change in dissovled oxygen 
in experimental unit $i$, 
$\beta_i$ and $\alpha_i$ are parameters of the PI curve as defined in equation \ref{eq:pi},
$\rho_i$ is total respiration of organisms in the sediment,
and $\epsilon_i$ is the residual error.
The residuals were modeld with Guassian distributions 
separated into observation- and mesocosm-level components
to account for repeated measurement of mesocosms across the two sampling days.
I modeled the effects of sediment treatment and experimental day as
%
\begin{equation} \label{eq:lm}
\begin{split}
\beta_i &\sim 1 + \text{day} \times \text{treatment} \\
\alpha_i &\sim 1 + \text{day} \times \text{treatment} \\
\rho_i &\sim 1 + \text{day} \times \text{treatment}
\end{split}
\end{equation}
%
using the ``formula'' notation for linear models, 
where ``1'' denotes the intercept,
and ``$\times$'' denotes both main effects and interactions.
Prior to fitting the model, I z-scored 
$\Delta \text{DO}_i$ and divided $\text{PAR}_i$ by the maximum observed PAR 
to improve numerical stability of the model fitting; 
model estimates were then back-scaled accordingly.
I fit the model with Stan run from R using the ``rstan'' package,
with Guassian(0, 1) priors for the linear predictor coefficients  
and Gamma(1.5, 3) priors for residual standard deviations;
these priors were weakly-informative relative to the scale of the standardized data.

To assess ecosystem engineering effects on different aspects primary production, 
I compared the full model to reduced versions 
that omitted main effects and interactions from the formulas 
of either $\beta_i$ and $\alpha_i$.
I removed main effects of sediment treatment and associated interactions with sample day 
together to reduce the number of models and because of the a prior expection that 
the treamtent effects were unlikely to remain constant through time.
Main effects and interactions associated with sediment treatment for $\rho_i$
were included in all models to reduce the number of alternatives and because 
previous analysis of these data 
found strong evidence for both the main effects and interactions on $\rho_i$.
I compared the alternative models using the information criterion associated 
with Leave-one-out cross validation (``LOOIC'') using the ``loo'' package.
LOOIC is conceptually similar to the  Akaike Information Criterion (AIC) 
and is closely related to the Widely Application Information Criterion (WAIC).
